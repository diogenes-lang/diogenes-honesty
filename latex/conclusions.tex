\section{Conclusions}\label{sec:conclusions}

% The tools presented in this paper 
Diogenes fills a gap between
foundational research on honesty~\cite{BTZ12sacs,BMSZ15jlamp,BZ15wsfm}
and more practical research on contract-oriented programming~\cite{CO2middleware}.
Its effectiveness can be improved in several ways,
ranging from the precision of the analysis,
to the informative quality of output messages provided by the honesty checkers.
The accuracy of the analysis could be improved \eg, by implementing the
type checking technique of~\cite{BSTZ13forte},
which can correctly classify the honesty of processes with delimitation/parallel 
within recursive processes.
For informativeness, 
% rather than outputting ``unsafe'' run-time configurations,
it would be helpful for programmers to know which parts of the program make it dishonest.

% The Eclipse plugin is still under development and we are considering to combine
% the text-based formats with one graphical editing framework. 
% e.g. Sirius~\cite{sirius-site} or Graphiti~\cite{graphiti-site}.

\hidden{The plugin supports the full \coco syntax and you can check
the honesty of any valid process (if decidable); however,
only a subset of processes can be translated to a java program, 
so the following examples are limited to a language's subset.}

% bart: ci sono gia' troppe self-references
% Regarding to our verification tool, we are trying to extend
% it to \emph{timed} session types~\cite{Bartoletti15forte}, 
% already supported by the contract-oriented middleware of \cite{CO2middleware}. 

\endinput

The plugin project should be improved and maintained. We are also considering to support multiple IDEs and browsers too, based on other coming soon Xtext features\footnote{\url{https://eclipse.org/Xtext/news.html} (visited on 2015-08)}.

%Moreover, the \coco language can evolve to get a better mapping to Java and other languages.

It would be also useful to consider different \coco target languages: in fact, any language that is supported by suitable high-level API can be chosen as target of a \coco specification. 
This could lead to a more general verification technique, considering an intermediate language with which verify the honesty (taking example from \cite{juhaszviper}).

Finally, the honesty verification technique can be further improved. So far, we do not support action types, due to some limitations of the middleware, and 
we cannot easily show which is the part of code that caused the culpability (in case of dishonest application). 
This is caused by the fact that, in order to check the honesty, our tool depends on an external one. The output is difficult to be understood by the user and cannot be easily parsed. 
This dependency also limits our tool in portability, requiring the user to take care of these dependencies.

