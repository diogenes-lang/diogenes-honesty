
\section{Background}\label{sec:background}


\subsection{Session types and compliance}
A contract describes the intended behaviour of \emph{one} of the two
participants involved in a session.  We use binary session
types~\cite{Honda98esop} to define contracts.  Session types are terms
of a process algebra featuring internal/external choice, and
recursion.

We assume a set of \emph{participants} (ranged over by
${\pmv A}, {\pmv B}, \ldots$), a set of \emph{branch labels} (ranged
over by $\atom{a}, \atom{b}, \ldots$), and a set of \emph{sorts}
ranged over by $\sortT, \sortTi, \ldots$ (e.g.\ \sort{int},
\sort{bool}, \sort{unit}).  Each sort $\sort{T}$ is populated by a set
of values, ranged over by $\valV, \valVi, \ldots$; as usual, we write
$\valV: \sortT$ to indicate that $\valV$ has sort~$\sortT$.

\begin{definition}[Contracts] \label{def:contracts:syntax}
Contracts are \emph{binary session types}, i.e.\ terms defined by the grammar:
\begin{align*}
%\text{Unilateral contracts} &&
    \contrP,\contrQ \;\; & ::= \;\;
    % \textstyle
    \SumInt[i \in \mathcal{I}]{\atomOut[i]{a} \sortT[i]}{\contrP[i]} \ \bnfmid \ 
    \SumExt[i \in \mathcal{I}]{\atomIn[i]{a} \sortT[i]}{\contrP[i]} \ \bnfmid \
    % \ready{\atomIn{a}\val{v}}.c \ \bnfmid \
    \rec{\contrX}{\contrP}
    \bnfmid \ \contrX
\end{align*}
where % we assume that
\begin{inlinelist} 
\item the index set $\mathcal{I}$ is finite,
\item \label{item:def:contracts:syntax:pairwise-distinct}
the labels $\atom[i]{a}$ in the prefixes of each summation are pairwise distinct, and 
\item recursion variables $\contrX$ are prefix-guarded.
\end{inlinelist}
\end{definition}

An internal sum $\SumInt[i]{\atomOut[i]{a}\sortT[i]}{\contrP[i]}$
allows a participant to choose one of the labels $\atom[i]{a}$, to
pass a value of sort $\sortT[i]$, and then to behave according to the
branch $\contrP[i]$.  Dually, an external sum
$\SumExt[i]{\atomIn[i]{a}\sortT[i]}{\contrP[i]}$ allows to wait for
the other participant to choose one of the labels $\atom[i]{a}$, and
then to receive a value of sort $\sortT[i]$ and behave according to
the branch $\contrP[i]$.
%
Empty internal/external sums are identified, and they are denoted with
$\cnil$, which represents a \emph{success state} wherein the
interaction has terminated.

%We use the (commutative and associative) binary operators to isolate a
%branch in a sum: \eg,\
%$\contrP = (\sumI{\atomOut{a}\sortT}{\contrPi}) \sumInt \contrPii$
%means that $\contrP$ has the form
%$\SumInt[i \in \mathcal{I}]{\atomOut[i]{a}\sortT[i]}{\contrP[i]}$ and
%there exists some $i \in \mathcal{I}$ such that
%$\sumI{\atomOut{a}\sortT}{\contrPi} =
%\sumI{\atomOut[i]{a}\sortT[i]}{\contrP[i]}$.
%Hereafter, we will omit the $\sort{unit}$ sort and the trailing
%occurrences of $\cnil$, and we will only consider contracts without
%free occurrences of recursion variables~$\contrX$.

To model the behaviour of two participants {\pmv A} and {\pmv B}
involved in a session, we compose their contracts together into
$\bic{\contrP}{\contrQ}$.  Their interaction is ruled by an
operational semantics \tizinote{citare} \Cref{citare}, where the two
participants alternate in firing actions: in partcular {\pmv A} can
fire either if she has an internal choice, or if she is committed to a
branch of an external choice.
%
To do that, the syntax of~\Cref{def:contracts:syntax} has been
extended with the term $\ready{\atomIn{a}\val{v}} \contrSeq
\contrP$,
which models a participant ready to input a value $\valV$ in a branch
with label $\atom{a}$, and then to continue as $\contrP$.  In other
words, $\ready{\atomIn{a}\val{v}}$ acts as a one-position buffer
shared between the two participants.

%
%
Two composed contracts may enjoy the property of
\emph{compliance}. The intuition is that if a contract $\contrP$ is
compliant with a contract $\contrQ$, then in all the configurations of
a computation of $\bic{\contrP}{\contrQ}$, whenever a participant
wants to choose a branch in an internal sum, the other participant
always offers the opportunity to do it.  Compliance guarantees
that % \emph{progress}:
whenever a computation of %$\bic{\contrP}{\contrQ}$ 
becomes stuck, then
both participants have reached the success state $\cnil$.


\subsection{The \coco calculus and honesty}\label{sec:co2}

We model agents and systems in the process calculus 
\coco\cite{BZ10lics,BTZ12coordination,BSTZ13forte},
which we instantiate with the contracts introduced in \Cref{sec:contract-session-types}.

Let $\vars$ and $\snames$ be disjoint sets
of \emph{variables} (ranged over by $x,y,\ldots$) and 
\emph{names} (ranged over by $s,t,\ldots$).
%
We assume a language of \emph{expressions}
(ranged over by $\expE, \expEi, \ldots$),
containing variables, values, and operators 
(\eg the usual arithmetic/logic ones).
The actual choice of operators is almost immaterial for the
subsequent technical development; here we just postulate
a function $\sem{\cdot}$ which maps (closed) expressions to values.
We assume that the sort of an expression is uniquely determined 
by the sorts of its variables.
We use $u,v,\ldots$ to range over $\vars \cup \snames$,
we use $\vec{u},\vec{v},\ldots$ to range over 
sequences of variables/names, and
$\vec{e}$ to range over sequences of expressions.
To make symbols lookup easier, we have summarised the syntactic categories 
and some notation in Table~\ref{def:notation}.
% Some of the symbols defined therein will only be used in later sections.

% \begin{table}[t]
% 	\footnotesize
% 	\hrulefill
% 	% \vspace{-10pt}
% 	\[
% 	\begin{array}{ll}
	
% 	\begin{array}{ll}
% 	\pmv{A}, \pmv{B}, \ldots & \text{Participant names}
% 	\\
% 	\atom{a}, \atom{b}, \ldots & \text{Branch labels}
% 	\\
% 	\sortT, \sortTi, \ldots & \text{Sorts}
% 	\\
% 	\valV, \valVi, \ldots & \text{Values}
% 	\\
% 	\contrP, \contrQ, \ldots & \text{Contracts}
% 	\\
% 	\gamma, \gammai, \ldots & \text{Contract configurations} 
% 	\\
% 	%   \contrP \compliant \contrQ & \text{Compliance}
% 	%   \\
% 	%   \cfrown{\pmv A}{\gamma} & \text{Culpability}
% 	%   \\
% 	\gamma \cmove{} \gammai & \text{Transition of contracts}
% 	\end{array}
	
% 	& \hspace{18pt}
	
% 	\begin{array}{ll}
% 	u, v,\ldots \mbox{\hspace{50pt}} & \text{Union of:} 
% 	\\
% 	s,t, \ldots \in \snames & \text{Session names} 
% 	\\
% 	x, y, \ldots \in \vars & \text{Variables} 
% 	\\
% 	\expE, \expEi, \ldots & \text{Expressions} 
% 	\\
% 	\procP,\procQ,\ldots & \text{Processes} 
% 	\\
% 	\sysS, \sysSi,\ldots & \text{Systems}
% 	\\
% 	%   \vabscontrP, \vabssysS, \ldots & \text{Value-abstract contracts/systems}
% 	%   \\
% 	%   \cabscontrP, \abssysS, \ldots & \text{Context-abstract contracts/systems}
% 	%   \\
% 	\sysS \pmove{} \sysSi & \text{Transition of systems}
% 	\end{array}
	
% 	\end{array}
% 	\]
% 	\hrulefill
% 	\vspace{-5pt}
% 	\caption{Summary of notation.} \label{def:notation}
% \end{table}


\begin{definition}\label{def:co2:syntax}
	The syntax of \coco is defined as follows:
	\[
	\small
	\begin{array}{r@{\hskip 0.1cm}lclcccccccccccc}   
	& \sysS \, (\text{Systems}) & ::= & 
	\emptysys 
	~\bnfmid ~ \sys {\pmv A} \procP 
	\; \bnfmid \; \sys s \gamma 
	\; \bnfmid \; (u)\sysS
	\; \bnfmid \; \sysS \mid \sysS
	\; \bnfmid \; \setenum{\freeze u \contrP}_{\pmv A}
	\\[.8pc]
	
	& \procP \, (\text{Processes})& ::= &  \textstyle 
	\cocoSum[i]{\pref[i] \cocoSeq \procP[\!i]}
	\; \bnfmid \; \ifte{\expE}{\procP}{\procP}
	\; \bnfmid \; \procX(\vec u,\vec e)
	\; \bnfmid \; (u)\procP
	\; \bnfmid \; \procP \cocoPar \procP
	\\[.8pc]
	
	& \pref \, (\text{Prefixes})& ::= & \tau
	\; \bnfmid \; \tell {} {\freeze u \contrP}
	\; \bnfmid \; \fact u {\atomOut{a} e}
	\; \bnfmid \; \fact u {\atomIn{a} x : \sortT}
	\; \bnfmid \; \ask {u} {\!\phi}
	\end{array}
	\]
	If $\vec u = u_0,\hdots,u_n$,
	we will use $(\vec u) \sysS$ and $(\vec u) \procP$ 
	as shorthands for $(u_0)\cdots(u_n) \sysS$ and $(u_0)\cdots(u_n) \procP$,
	respectively. %
	We also assume the following syntactic constraints on processes and systems:
	\begin{enumerate}
		
		\item each occurrence of named processes is prefix-guarded;
		
		\item in $(\vec u)(\sys {\pmv A} \procP \mid \sys{\pmv B} \procQ \mid \cdots)$,
		it must be $\pmv A \neq \pmv B$;
		
		\item in $(\vec u)(\sys s \gamma \mid \sys t \gammai \mid \cdots)$,
		it must be $s \neq t$.
		
	\end{enumerate}
\end{definition}


\begin{figure}[t]
	\hrulefill
	\footnotesize
	\begin{center}
		commutative monoidal laws for $\mid$ on processes and systems
	\end{center}
	\vspace{-10pt}
	\[
	\begin{array}{c}
	\sys {\pmv A} {(v) \procP} \equiv \sys{(v) \, {\pmv A}} \procP 
	\hspace{20pt}
	\sysFmt{Z} \mid (u) \sysFmt{Z'} \equiv (u)(\sysFmt{Z} \mid \sysFmt{Z'}) 
	\;\;\text{if}\ u \not\in \fv{\sysFmt{Z}} \cup \fn{\sysFmt{Z}}
	\\[8pt]
	(u)(v) \sysFmt{Z} \equiv (v)(u) \sysFmt{Z}
	\hspace{20pt}
	(u) \sysFmt{Z} \equiv \sysFmt{Z}
	\;\;\text{if}\ u \not\in \fv{\sysFmt{Z}} \cup \fn{\sysFmt{Z}}
	\hspace{20pt} 
	\setenum{\freeze s \contrP}_{\pmv A} \equiv \pnil 
	\end{array}
	\]
	\hrulefill
	\vspace{-5pt}
	\caption[Structural equivalence for \coco]{Structural equivalence for \coco 
		($\sysFmt{Z},\sysFmt{Z'}$ range over systems or processes).} \label{fig:co2:equiv}
	\vspace{-10pt}
\end{figure}


\emph{Systems} $\sysS, \sysSi,\ldots$ are the parallel composition of 
\emph{participants} $\sys {\pmv A} \procP$,
\emph{sessions} $\sys s \gamma$,
\emph{delimited systems} $(u)\sysS$, 
and \emph{latent contracts} $\setenum{\freeze{u\!\!}{\contrP}}_{\pmv A}$.
A latent contract $\setenum{\freeze{x\!\!}{\contrP}}_{\pmv A}$ 
represents a contract $\contrP$ (advertised by {\pmv A}) which
has not been stipulated yet; upon stipulation, the variable $x$ will be
instantiated to a fresh session name. 
% Latent contracts of the form 
% $\setenum{{\freeze s c}_{\pmv A}}$, where $s$ is a session \emph{name} are discarded
% by the axiom $\setenum{{\freeze s c}_{\pmv A}} \equiv \pnil$ in~\Cref{fig:co2:equiv}.

\emph{Processes} $\procP, \procQ, \ldots$ are
prefix-guarded (finite) sums of processes,
conditionals $\ifte{\expE}{\procP}{\procQ}$
(where $\expE$ is a boolean valued expression),
named processes $\procX(\vec u,\vec e)$ %
(used \eg\ to specify recursive behaviours),
delimited processes $(u) \procP$,
and parallel compositions $\procP \cocoPar \procP$.

\emph{Prefixes} $\pref$ include silent action $\tau$, 
contract advertisement $\tell{}{\freeze u \contrP}$, 
output action $\fact{u}{\atomOut{a}\expE}$,
input action $\fact{u}{\atomIn{a}x:\sortT}$,
and contract query $\ask{u}{\phi}$
(where $\phi$ is an LTL formula on $\gamma$).
%
In each prefix $\pref \neq \tau$, 
the index $u$ refers to the target session involved in
the execution of $\pref$.

\smallskip
The only binder for names is the
delimitation $(u)$, both in systems and processes.
Instead, variables have two binders:
delimitations $(x)$ (both in systems and processes),
and input actions.
Namely, in a process $\fact u {\atomIn{a}x}:\sortT.\, \procP$, 
the variable $x$ in the prefix binds the occurrences of $x$ within $\procP$.
Note that ``value-kinded'' variables in input actions 
will be replaced by values,
while ``name-kinded'' variables used in delimitations 
will be replaced by session names.
Accordingly, we avoid confusion between these two kinds of variables.
For instance, we forbid
$\fact{u}{\atomIn{a}x}.\, \fact{x}{\atomOut{b}{\valV}}$
and
$(x) \, \fact{u}{\atomOut{a}{x}}$.
%
% We assume that the variables used in input actions are disjoint from 
% those used in delimitations.

Free \emph{session} names/variables in a prefix are defined as follows:
$\fnv{\tau} = \emptyset$, and
\(
\fnv{\tell{}{\freeze u \contrP}} = 
\setenum{u} = 
\fnv{\fact u {\atomOut a}\expE} =
\fnv{\fact u {\atomIn a}x:\sortT}
\).
Free variables/names of systems/processes are defined accordingly, 
and they are denoted by $\fv{}$ and $\fn{}$.
A system or a process is \emph{closed} when it has no free variables.

We write $\pref[1] \cocoSeq \procP[1] \cocoPlus \pref[2] \cocoSeq
\procP[2]$ for $\cocoSum[{i \in \setenum{1,2}}] {\pref[i]} \cocoSeq
\procP[i]$, and $\pnil$ for $\cocoSum[{\emptyset}]\procP$.
%
We stipulate that each process identifier $\procX$ 
has a unique defining equation
$\procX(x_1, \ldots, x_j) \mmdef \procP$ such that $\fv{\procP} \subseteq
\setenum{x_1,\ldots,x_j} \subseteq \vars$.
We will sometimes omit %
the arguments of $\procX(\vec u, \vec e)$ when they are clear from the context.
As usual, we omit trailing occurrences of~$\pnil$ in processes.

We call  \emph{obligations} those actions a participant $\pmv A$ at a
session $s$ in $\sysS$ has to fire.  
%
A \coco process is \emph{honest} if, in all possible interactions, it
always fulfils its contractual obligations.  

%
Verifying honesty in \coco is undecidable in
general~\cite{BTZ12coordination}, because one must consider \emph{all}
possible execution contexts, which are infinite.  Even with usual
syntactic restrictions required to make processes finite-state (\eg\
no delimitation/parallel under process definitions) value-passing
makes the semantics of \coco infinite-state.

Even thought, adopting the aforementioned restrictions and with an abstraction
to approximate values, a checker for honesty has been  implemented
in Maude \cite{Maude01} and described in details in \cite{verifiable}.


\subsection{The contract oriented middleware}

\coco middleware  is a \textit{contract-oriented middleware} \cite{CO2middleware}
that aims to monitor the interaction between mutually distrusting
services and simplify the development of distributed applications.
%
A service can \textit{advertise} its contract without worrying about
the search of a \textit{compliant} peer to interact with. The
middleware carry about the creation of a session and monitors the
involved services to detect contract violations.

In order to interact with the middleware, a developer can choose
between the RESTFUL API, or language specific API.

\tizinote{da spostare avanti}
The API allows us to advertise a contract only as plain Java
\incodeType{String} in two formats: XML and \textit{timed
  session-types} \cite{Bartoletti15forte}. The former is too verbose
and does not strictly comply our session-types definition (see
\Cref{def:contracts:syntax}); the latter is a superset of our
specification, so we prefer to use it on communicating with the
middleware.
In \Cref{chap:co2-to-java} we show an extended version of these API,
providing a Java representation for contracts.

