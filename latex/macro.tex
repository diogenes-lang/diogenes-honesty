\newcommand{\HSL}[1][HIC SUNT LEONES]{\vbox{\medskip\noindent\hrulefill \\[5pt]
  \rule{1ex}{1ex}\hspace{\stretch{1}} \; {#1}\hspace{\stretch{1}}\rule{1ex}{1ex} \\ \smallskip\noindent\hrulefill \\}}

\renewcommand{\vec}[1]{\ensuremath{\textit{\textbf{#1}}}}
\newcommand{\relR}{\mathcal{R}}

\newcommand{\codefont}{\fontsize{10}{10}\selectfont}
\newcommand{\code}[1]{{\tt\codefont {#1}}}

% for braces in alltt mode
\newcommand{\braceleft}{\symbol{`\{}}
\newcommand{\braceright}{\symbol{`\}}}
\newcommand{\angleleft}{\symbol{`\<}}
\newcommand{\angleright}{\symbol{`\<}}
% \newcommand{\mytilde}{\scalebox{0.85}{\url{~}}}
\newcommand{\mytilde}{\raisebox{-3pt}{{\symbol{`\~}}}}
\newcommand{\mydash}{\raisebox{0pt}{{\symbol{`\-}}}}

\newcommand\bslash{\symbol{`\\}}
\def\etc{etc.\@\xspace}
\newcommand{\eg}{e.g.\@\xspace}
\newcommand{\ie}{i.e.\@\xspace}
\newcommand{\wrt}{w.r.t.\@\xspace}

\newenvironment{proof}{\noindent{\it Proof.}}{}
\newenvironment{proofof}[2]{{#2}\begin{proof}}{\end{proof}}

\newcounter{fact}

%% NOTE: some theorem styles are already defined in llncs.cls
\declaretheorem[numberwithin=section]{theorem}
\declaretheorem[numberlike=theorem]{definition}
\declaretheorem[numberlike=theorem]{notation}
\declaretheorem[numberlike=theorem]{lemma}
\declaretheorem[numberlike=theorem]{example}
\declaretheorem[numberlike=theorem]{corollary}
\declaretheorem[numberlike=theorem]{remark}
\declaretheorem[numberlike=theorem]{proposition}

\declaretheorem[name=Definition,numberlike=theorem]{appdefinition} 
\declaretheorem[name=Theorem,numberlike=theorem]{apptheorem} 
\declaretheorem[name=Lemma,numberlike=theorem]{applemma} 

\newcommand{\qedhere}{}


\def\finex{{\unskip\nobreak\hfil
\penalty50\hskip1em\null\nobreak\hfil$\blacklozenge$
\parfillskip=0pt\finalhyphendemerits=0\endgraf}}

\newcommand{\rname}[1]{(\mbox{\sc #1})}

% \definecolor{shadecolor}{rgb}{1,0.99,0.9}
\def\act#1#2{#1 @ #2}

\newcommand{\Real}[1]{\mathrm{Real}}

\newcommand{\coco}{CO\textsubscript{2}\xspace}
% \newcommand{\cccp}{\mbox{\texttt{adcc}}}

\newcommand{\tuple}[1]{\ensuremath{\langle {#1} \rangle}}
\newcommand{\hide}[2]{{#1} \setminus {#2}}

\newcommand{\names}{\ensuremath{\mathcal{N}}}
% WARNING: modificato rispetto a sacs
% \newcommand{\snames}{\names_S}
\newcommand{\snames}{\names}
\newcommand{\pnames}{\names_P}

\newcommand{\vars}{\mathcal V}
\newcommand{\svars}{\vars_S}
\newcommand{\pvars}{\vars_P}

\def\colorPtp{\color{ForestGreen}}
% \newcommand{\pmv}[1]{\colorPtp{\ensuremath{\mathsf{#1}}}}
\newcommand{\pmv}[1]{\ensuremath{\mathsf{\colorPtp{#1}}}}
\newcommand{\pmvset}[1]{\ensuremath{\mathcal{\colorPtp{#1}}}}
\newcommand{\Part}{\boldsymbol{\pmvset{P}}}

\newcommand{\Atom}{\textup{\textbf{A}}}
\newcommand{\dummy}{\varepsilon}
\newcommand{\TSname}{\ensuremath{\operatorname{TS}}}
\newcommand{\TS}[1]{\TSname({#1})}
\newcommand{\absTSname}{\TSname_{\pmv A}}
\newcommand{\absTS}[1]{\absTSname({#1})}
\newcommand{\vabsTSname}{\TSname_{\expDummy}}
\newcommand{\vabsTS}[1]{\vabsTSname({#1})}
\newcommand{\Paths}[1]{\operatorname{Paths}({#1})}


\newcommand{\expE}{\mathit{e}}
\newcommand{\expEi}{\expE'}
\newcommand{\expDummy}{\star}

% CO2 syntax

\def\cocoColor{\color{MidnightBlue}}
\newcommand{\cocoFmt}[1]{{\cocoColor{\code{#1}}}}

\newcommand{\true}{\cocoFmt{true}}
\newcommand{\false}{\cocoFmt{false}}
\newcommand{\cond}{\cocoFmt{if}}
\newcommand{\then}{\cocoFmt{then}}
\newcommand{\owise}{\cocoFmt{else}}
\newcommand{\ifte}[3]{\cond~{#1}~\then~{#2}~\owise~{#3}}
\newcommand{\fact}[2]{\cocoFmt{do}_{#1}\,{#2}}
\newcommand{\tell}[2]{\cocoFmt{tell}^{#1}\,{#2}}
\newcommand{\freeze}[2]{\downarrow_{#1}{#2}}
\newcommand{\ask}[2]{\ifempty{#2}{\cocoFmt{ask}_{{#1}}}{\cocoFmt{ask}_{{#1}}\,{#2}}}
\newcommand{\fuse}{\cocoFmt{fuse}}
\newcommand{\pref}[1][]{\cocoFmt{\pi}_{{\cocoColor{#1}}}}
\newcommand{\prefi}[1][]{\cocoFmt{\pi'}_{{\cocoColor{#1}}}}
\let\greektau\tau
\renewcommand{\tau}{\cocoFmt{\greektau}}
% \newcommand{\checkp}[1]{\mathsf{check}\,#1}
% \newcommand{\says}{\ensuremath{\;\mathit{says}\;}}
\newcommand{\says}{\ensuremath{:}}
\newcommand{\psays}{\ensuremath{\;\mathit{:}\;}}
\newcommand{\cocoSeq}{\mathbin{\!{\cocoColor{.}}\!}}
\newcommand{\cocoPlus}{\mathbin{{\cocoColor{+}}}}
\newcommand{\cocoSum}[2][]{\mathord{{\cocoColor{\sum_{#1}{#2}}}}}
\newcommand{\cocoPar}{\mathbin{{\cocoColor{\mid}}}}

% Systems

\def\sysColor{\color{MidnightBlue}}
\newcommand{\sys}[2]{{#1} [{#2}] }
\newcommand{\sysFmt}[1]{{\sysColor{#1}}}
\newcommand{\sysS}[1][]{\mathord{\sysFmt{S}_{\sysColor{#1}}}}
\newcommand{\sysSi}[1][]{\mathord{\sysColor{\sysS'_{#1}}}}
\newcommand{\sysSii}[1][]{\mathord{\sysColor{\sysS''_{#1}}}}
\newcommand{\sysSiii}[1][]{\mathord{\sysColor{\sysS'''_{#1}}}}
\newcommand{\sysNil}{\sysFmt{\mathbf{0}}}
\newcommand{\emptysys}{\sysNil}


\newcommand{\vabscontrP}[1][]{\contrFmt{\hat{\contrP[#1]}}}
\newcommand{\vabscontrPi}[1][]{\contrFmt{\hat{\contrP[#1]}\contrColor{'}}}
\newcommand{\vabscontrPii}[1][]{\contrFmt{\hat{\contrP[#1]}\contrColor{''}}}
\newcommand{\vabscontrQ}[1][]{\contrFmt{\hat{\contrQ[#1]}}}
\newcommand{\vabscontrQi}[1][]{\contrFmt{\hat{\contrQ[#1]}\contrColor{'}}}
\newcommand{\vabscontrQii}[1][]{\contrFmt{\hat{\contrQ[#1]}\contrColor{''}}}

\newcommand{\vabssysS}[1][]{\sysFmt{\hat{\sysS}_{#1}}}
\newcommand{\vabssysSi}[1][]{\sysFmt{\hat{\sysS}_{#1}'}}
\newcommand{\vabssysSii}[1][]{\sysFmt{\hat{\sysS}_{#1}''}}
\newcommand{\vabssysSiii}[1][]{\sysFmt{\hat{\sysS}_{#1}'''}}

\newcommand{\cabscontrP}[1][]{\contrFmt{\tilde{\contrP[#1]}}}
\newcommand{\cabscontrPi}[1][]{\contrFmt{\tilde{\contrP[#1]}\contrColor{'}}}
\newcommand{\cabscontrQ}[1][]{\contrFmt{\tilde{\contrQ[#1]}}}
\newcommand{\cabscontrQi}[1][]{\contrFmt{\tilde{\contrQ[#1]}\contrColor{'}}}
\newcommand{\cabsprocP}[1][]{\procFmt{\tilde{\procP[#1]}}}
\newcommand{\cabsprocPi}[1][]{\procFmt{\tilde{\procPi[#1]}}}
\newcommand{\cabsprocQ}[1][]{\procFmt{\tilde{\procQ[#1]}}}
\newcommand{\cabsprocQi}[1][]{\procFmt{\tilde{\procQi[#1]}}}

\newcommand{\abssysS}[1][]{\sysFmt{\tilde{\sysS[#1]}}}
\newcommand{\abssysSi}[1][]{\sysFmt{\tilde{\sysS[#1]}{}'}}
\newcommand{\abssysSii}[1][]{\sysFmt{\tilde{\sysS[#1]}{}''}}
\newcommand{\abssysSiii}[1][]{\sysFmt{\tilde{\sysS[#1]}{}'''}}


% Processes

\def\procColor{\color{MidnightBlue}}
\newcommand{\procFmt}[1]{{\procColor{#1}}}
\newcommand{\procP}[1][]{\mathord{\procFmt{P}_{\procColor{#1}}}}
\newcommand{\procPi}[1][]{\mathord{\procP[#1]\procColor{'}}}
\newcommand{\procPii}[1][]{\mathord{\procP[#1]\procColor{''}}}
\newcommand{\procQ}[1][]{\mathord{\procFmt{Q}_{\procColor{#1}}}}
\newcommand{\procQi}[1][]{\mathord{\procQ[#1]\procColor{'}}}
\newcommand{\procQii}[1][]{\mathord{\procQ[#1]\procColor{''}}}
\newcommand{\procR}[1][]{\mathord{\procFmt{R}_{\procColor{#1}}}}
\newcommand{\procRi}[1][]{\mathord{\procR[#1]\procColor{'}}}
\newcommand{\procRii}[1][]{\mathord{\procR[#1]\procColor{''}}}
\newcommand{\procX}[1][]{\operatorname{{\procColor{\procFmt{X}_{#1}}}}}
\newcommand{\procY}[1][]{\mathord{\procFmt{Y}_{\procColor{#1}}}}
\newcommand{\procNil}{\procFmt{\mathbf{0}}}
\newcommand{\pnil}{\procNil}


%%% If/then check for empty strings (requires xifthen for \ifempty)
\newcommand{\ifempty}[3]{%
  \ifthenelse{\isempty{#1}}{#2}{#3}%
}

% \newcommand{\sep}{\,\bnfmid\,}

\newcommand{\redrule}[2]{
\prooftree
{#1}
\justifies
{#2}
\endprooftree
}
\newcommand{\thaw}{\uparrow}
\newcommand{\agreement}[4]{{#1} \vartriangleright_{#3}^{#4} {#2}}

\newcommand{\dmid}{\mid\hspace{-0pt}\mid}


%%% General macros
\newcommand{\bla}{bla, bla, bla...}
\newcommand{\compile}[2]{\ifthenelse{\equal{#1}{yes}}{#2}{}}
\newcommand{\hidden}[1]{}

\newcommand{\cf}[2]{
  \fontsize{#1}{#1}{\selectfont{#2}}
}

\newcommand{\namedef}[2]
{\begin{center}\begin{tabular}{lr}
      {$#2$}
      & \qquad
      {\emph{#1}}
    \end{tabular}\end{center}
}
% \newcommand{\max}[1]{\operatorname{max}({#1})}
% \newcommand{\min}[1]{\operatorname{cod}({#1})}

\newcommand{\dom}[1]{\mathrm{dom}({#1})}
\newcommand{\seq}{\stackrel{\cdot}{=}}

\newcommand{\zero}{\mathbf{0}}
\newcommand{\fn}[1]{\ifempty{#1}{\operatorname{fn}}{\operatorname{fn}(#1)}}
\newcommand{\bn}[1]{\ifempty{#1}{\operatorname{bn}}{\operatorname{bn}(#1)}}
\newcommand{\fv}[1]{\ifempty{#1}{\operatorname{fv}}{\operatorname{fv}(#1)}}
\newcommand{\bv}[1]{\ifempty{#1}{\operatorname{bv}}{\operatorname{bv}(#1)}}
\newcommand{\fnv}[1]{\ifempty{#1}{\operatorname{fnv}}{\operatorname{fnv}(#1)}}
\newcommand{\mmdef}{\mbox{$\;\stackrel{\textrm{\tiny def}}{=}\;$}}



\newcommand{\irule}[2]{
  \begin{array}{c}
    #1  \\ \hline
    #2
  \end{array}}
\newcommand{\res}{\mbox{$\nu \,$}}
\newcommand{\trd}[2]{\mbox{$[\hspace{-0.65mm}[ \; #1 \; ]\hspace{-0.65mm}]_{#2}$}}
\newcommand{\trdp}[3]{\mbox{$[\hspace{-0.65mm}[ \; #1 \; ]\hspace{-0.65mm}]_{#2}^{#3}$}}



\newcommand{\ori}[1]{\mathbf{#1}}
% \newcommand{\freq}{f\delta}


\def \overunderstackrel#1#2#3{\mathrel{\mathop{#1}\limits^{#2}_{#3}}}
\def \overstackrel#1#2{\mathrel{\mathop{#1}\limits^{#2}}}
\def \understackrel#1#2{\mathrel{\mathop{#1}\limits_{#2}}}


\newcommand{\trans}[3]{{#1} \stackrel{{#2}}{\rightarrow} {#3}}
% \newcommand{\trans}[3]{\setbox0=\hbox{$\ {}^{#2}\ $}
%   \setbox1=\hbox{$\longrightarrow$}
%   \ifdim\wd0<\wd1\setbox0=\box1\else\relax\fi
%   \mbox{${#1}\,\mathop{\hbox to \wd0{\rightarrowfill}}\limits^{#2}\,{#3}$}
% }
% \newcommand{\Set}{\mbox{\bf Set}}
% \newcommand{\Fun}{\mbox{\bf Fun}}


\newcommand{\subs}[2]{\{\nicefrac{#1}{#2}\}}

\newcommand{\sidebyside}[2]{
  \begin{tabular}{ll}
    \begin{minipage}{.5\linewidth} {#1}  \end{minipage}
    &
    \begin{minipage}{.5\linewidth} {#2}  \end{minipage}
  \end{tabular}
}


%%% MACRO-PCL

\newcommand{\bnfdef}{::=}
\newcommand{\bnfmid}{\;\big|\;}

% \newcommand{\coimp}{\twoheadrightarrow}
\newcommand{\imp}{\rightarrow}
\newcommand{\nrule}[1]{{\scriptsize \textsc{#1}}}
\newcommand{\smallnrule}[1]{{\tiny \textsc{#1}}}
% \newcommand{\irule}[2]{\frac{\textstyle\rule[-1.3ex]{0cm}{3ex}#1}%
% {\textstyle\rule[-.5ex]{0cm}{3ex}#2}}
\newcommand{\sem}[2][]{\mbox{\ensuremath{\llbracket{#2}\rrbracket_{#1}}}}

\newcommand{\pcl}{\textup{PCL\;}}
\newcommand{\pclsays}{\ensuremath{\,\pcl^{\!\!\!\says}}}
\newcommand{\pclminus}{\ensuremath{\,\pcl^{\!\!\!-}}}

\newcommand{\rew}{\rightarrow}

% \newcommand{\powset}[1]{\wp(#1)}
\newcommand{\powset}[1]{2^{#1}}
\newcommand{\bind}[2]{\nicefrac{#2}{#1}}
\newcommand{\setenum}[1]{\{#1\}}
\newcommand{\setcomp}[2]{\left\{{#1} \,\middle|\, {#2}\right\}}
% \newcommand{\setcomp}[2]{\{{#1} \,\mid\, {#2}\}}
\newcommand{\entails}{\vdash}

\newcommand{\lbl}[2]{\pmv{#1} \says \, {#2}}

%% MACRO Contratti Castagna

\let\greekgamma\gamma
\def\contrColor{\color{Plum}}
\newcommand{\contrFmt}[1]{{\contrColor{#1}}}

\newcommand{\bang}{{\contrColor{\textup{\texttt{\symbol{`\!}}}}}}
\newcommand{\qmark}{{\contrColor{\textup{\texttt{\symbol{`\?}}}}}}

\newcommand{\atom}[2][]{\contrFmt{\ifempty{#1}{{\code{#2}}}{{\code{#2}}_{#1}}}}
% \newcommand{\atomL}[1][]{\ifempty{#1}{\ell}{\ell_{#1}}} % deprecated
\newcommand{\atomIn}[2][]{\atom[#1]{#2}{\qmark}}
\newcommand{\atomOut}[2][]{\atom[#1]{#2}{\bang}}

\newcommand{\sort}[2][]{\contrFmt{\ifempty{#1}{{\code{#2}}}{{\code{#2}}_{#1}}}}
\newcommand{\sortT}[1][]{\sort[#1]{T}}
\newcommand{\sortTi}[1][]{\contrFmt{\sort[#1]{T}'}}
\newcommand{\val}[2][]{\contrFmt{\ifempty{#1}{{\code{#2}}}{{\code{#2}}_{#1}}}}
\newcommand{\valV}[1][]{\val[#1]{v}}
\newcommand{\valVi}[1][]{\contrFmt{\val[#1]{v}'}}

\renewcommand{\gamma}[1][]{\mathord{\contrFmt{\greekgamma}_{\contrFmt{#1}}}}
\newcommand{\gammai}[1][]{\mathord{\gamma[#1]\contrColor{'}}}
\newcommand{\gammaii}[1][]{\mathord{\gamma[#1]\contrColor{''}}}
\newcommand{\gammaiii}[1][]{\mathord{\gamma[#1]\contrColor{'''}}}
\newcommand{\vabsgamma}[1][]{\contrFmt{\hat{\gamma[#1]}}}
\newcommand{\vabsgammai}[1][]{\contrFmt{\hat{\gamma[#1]}\contrColor{'}}}
\newcommand{\contr}[1]{\contrFmt{#1}}
\newcommand{\contrP}[1][]{\mathord{\contrFmt{c}_{\contrColor{#1}}}}
\newcommand{\contrPi}[1][]{\mathord{\contrFmt{c'}_{\!\contrColor{#1}}}}
\newcommand{\contrPii}[1][]{\mathord{\contrFmt{c''}_{\!\!\contrColor{#1}}}}
\newcommand{\contrPiii}[1][]{\mathord{\contrFmt{c'''}_{\!\!\!\contrColor{#1}}}}
\newcommand{\contrQ}[1][]{\mathord{\contrFmt{d}_{\contrColor{#1}}}}
\newcommand{\contrQi}[1][]{\mathord{\contrQ[#1]\contrColor{'}}}
\newcommand{\contrQii}[1][]{\mathord{\contrQ[#1]\contrColor{''}}}
\newcommand{\contrQiii}[1][]{\mathord{\contrQ[#1]\contrColor{'''}}}
\newcommand{\contrR}[1][]{\mathord{\contrFmt{r}_{\contrColor{#1}}}}
\newcommand{\contrRi}[1][]{\mathord{\contrR[#1]\contrColor{'}}}
\newcommand{\contrRii}[1][]{\mathord{\contrR[#1]\contrColor{''}}}
\newcommand{\contrX}[1][]{\mathord{\contrFmt{X}_{\contrColor{#1}}}}
\newcommand{\contrXi}[1][]{\mathord{\contrX[#1]\contrColor{'}}}
\newcommand{\contrY}[1][]{\mathord{\contrFmt{Y}_{\contrColor{#1}}}}
\newcommand{\contrYi}[1][]{\mathord{\contrY[#1]\contrColor{'}}}
\newcommand{\contrZ}[1][]{\mathord{\contrFmt{Z}_{\contrColor{#1}}}}
\newcommand{\contrZi}[1][]{\mathord{\contrZ[#1]\contrColor{'}}}

% N-ary (big) operators
\newcommand{\SumIntRaw}[1]{\mathop{\contrColor{\bigoplus_{#1}}}}
\newcommand{\SumExtRaw}[1]{\mathop{\contrColor{\sum_{#1}}}}

% Binary (small) operators
\newcommand{\sumInt}{\mathbin{\contrColor{\oplus}}}
\newcommand{\sumExt}{\mathbin{\contrColor{+}}}

% Sequencing
\newcommand{\contrSeq}{\mathbin{\contrColor{.}}}

% Non-singleton sums (optional argument #1 is the index set)
\newcommand{\SumInt}[3][]{\SumIntRaw{#1} {#2} \contrSeq {#3}}
\newcommand{\SumExt}[3][]{\SumExtRaw{#1} {#2} \contrSeq {#3}}

% Singleton sums
\newcommand{\sumI}[2]{{#1} \contrSeq {#2}}
\newcommand{\sumE}[2]{{#1} \contrSeq {#2}}

% Failure
\newcommand{\contrFail}{\contrFmt{\code{fail}}}

% Ready
\newcommand{\ready}[1]{\ifempty{#1}{\contrFmt{\code{rdy}}}{\contrFmt{\code{rdy}}\; {#1}}}
\newcommand{\rec}[2]{\contrFmt{\operatorname{\code{rec}}}\,{\contrFmt{#1}}\! \contrSeq {\contrFmt{#2}}}
\newcommand{\contrNil}{\contrFmt{\code{0}}}
\newcommand{\cnil}{\contrNil}

\newcommand{\E}{\textit{E}}
\newcommand{\co}[1]{{\operatorname{co}}\!\left({#1}\right)}
\newcommand{\rs}[1]{{\operatorname{RS}}\!\left({#1}\right)}
\newcommand{\rdyname}{\operatorname{rdy}}
\newcommand{\rdy}[1]{\ifempty{#1}{\rdyname}{{\rdyname}\!\left({#1}\right)}}
\newcommand{\rd}[3]{{{\operatorname{RD}}^{#1}_{#2}}({#3})}
\newcommand{\WRD}{\operatorname{WRD}}
\newcommand{\wrd}[3]{{{\WRD}^{#1}_{#2}}({#3})}
\newcommand{\obbl}[3]{{{\operatorname{O}}^{#1}_{#2}}({#3})}
\newcommand{\RdyS}[2]{\operatorname{Rdy}^{#1}_{#2}}

\newcommand{\abs}{\alpha}
\def\cabsColor{\color{Black}}
\newcommand{\cabsFmt}[1]{{\cabsColor{#1}}}
\newcommand{\cabs}[2][]{\ifempty{#2}{\abs_{\pmv{#1}}}{\abs_{\pmv{#1}}({#2})}}

\def\vabsColor{\color{Black}}
\newcommand{\vabsFmt}[1]{{\vabsColor{#1}}}
\newcommand{\vabsDummy}{\ensuremath{\vabsFmt{\star}}}
\newcommand{\vabs}[2][]{\vabsFmt{\ifempty{#2}{\abs_{#1}^{\scalebox{0.8}{\vabsDummy}}}{\abs_{#1}^{\scalebox{0.8}{\vabsDummy}}({#2})}}}
\newcommand{\vabsatom}[2][]{\ifempty{#1}{{\vabsFmt{#2}}}{{\vabsFmt{#2}}_{\vabsFmt{#1}}}}
\newcommand{\vabsatomA}[1][]{\vabsatom[#1]{a}}
\newcommand{\vabsatomB}[1][]{\vabsatom[#1]{b}}
\newcommand{\vabsRdyS}[2]{\vabs{}\text{-}\RdyS{#1}{#2}}

\newcommand{\cvabs}[2][]{\abs_{\pmv #1}^{\scalebox{0.8}{\vabsDummy}}({#2})}

\newcommand{\absRdyS}[2]{\abs\text{-}\RdyS{#1}{#2}}

\newcommand{\compliant}[0]{\bowtie}
\newcommand{\dual}[1]{\operatorname{dual}\!\left(#1\right)}
\newcommand{\ncompliant}[0]{\not\bowtie}
\newcommand{\subc}[0]{\sqsubseteq}

% \newcommand{\cmove}[1]{\xrightarrow{#1}_{\bf C}}
\newcommand{\cmove}[1]{\mathrel{\xrightarrow{#1}\hspace{-1.8ex}\rightarrow}}
\newcommand{\pmove}[2][]{\xrightarrow{#2}}

\newcommand{\cabscmove}[1]{\mathrel{\cmove{#1}_{\pmv A}}}
\newcommand{\cabspmove}[2][]{\mathrel{{\xrightarrow{#2}_{\pmv A}^{#1}}}}
\newcommand{\abscmove}[1]{\cabscmove{#1}}
\newcommand{\abspmove}[2][]{\cabspmove{#1}}

\def\abscmovectx#1{\abscmove{\ctx \says {#1}}}

\newcommand{\vabscmove}[1]{\mathrel{\cmove{#1}_{\scalebox{0.8}{\vabsDummy}}}}
\newcommand{\vabspmove}[1]{\mathrel{\pmove{#1}_{\scalebox{0.8}{\vabsDummy}}}}

\newcommand{\csmiley}[3][]{{#2} \hspace{3pt}\dot{}\hspace{4pt}\dot{}\hspace{-6pt}{\smallsmile}\hspace{1pt}_{{#1}} {#3}}
\newcommand{\cfrown}[3][]{{#2} \hspace{3pt}\dot{}\hspace{4pt}\dot{}\hspace{-6pt}{\smallfrown}\hspace{1pt}_{{#1}} {#3}}

%%% Bilateral contracts

\newcommand{\pbic}[4]{\ifempty{#1}{{#2}, {#4}}{{\pmv {#1}} \says{#2} \mid {\pmv {#3}} \says {#4}}}
\newcommand{\bic}[2]{\pbic{\pmv A}{#1}{\pmv B}{#2}}

% \newcommand{\pbisb}[4]{\ifempty{#1}{{#2} \mmid {#4}}{{\pmv {#1}}[{#2}] \mmid {\pmv {#3}}[{#4}]}}
% \newcommand{\bisb}[2]{\pbisb{}{#1}{}{#2}}

%%% Label CO2

\newcommand{\does}[3][]{{#2} \psays \fact{#1}{#3}}
\newcommand{\del}[2]{\mathrm{del}_{#1}({#2})}

%%% Labels abstract LTS

\newcommand{\unblocked}{\tau}
\newcommand{\mayblock}{?}
\newcommand{\ctx}{{\colorPtp{\mathit{ctx}}}}

\newcommand{\readydo}[2]{\textit{RD}_{#1}({#2})}
\newcommand{\unblocks}[2]{{#1} \;\operatorname{unblocks}\; {#2}}
\newcommand{\realizes}[4][]{{#2} \models_{#4}^{#1} {#3}}
\newcommand{\notrealizes}[4][]{{#2} \not\models_{#4}^{#1} {#3}}
\newcommand{\canonical}{\text{$\abs$-honest}}
\newcommand{\canonicity}{\text{$\abs$-honesty}}
\newcommand{\appr}{{\prec}_{\pmv A}}

\newcommand{\vabsentails}{\mathrel{\vdash_{\expDummy}}}
\newcommand{\cabsentails}{\mathrel{\vdash}_{\pmv A}}
\newcommand{\cabsentailsctx}{\mathrel{\vdash}_{\ctx}}

\newcommand{\concr}{\gamma}
%\newcommand{\con}{\gamma}
\newcommand{\cta}[1]{\abs_{\pmv A} (#1)}
\newcommand{\atc}[1]{\concr_{\pmv B} (#1)}
%\newcommand{\atc}[1]{\con_{\pmv A} (#1)}
\newcommand{\abswrd}[3]{{{\abs\textit{-WRD}}^{#1}_{#2}}(#3)}
\newcommand{\absready}{\abs{\text{-ready}}}

%%% Rimozione delimitazioni

\newcommand{\open}[1]{open(#1)}

%%% MACRO per enunciati e dimostrazioni in appendice

\newenvironment{barttheorem}[2][]{\noindent\textbf{{#2}{#1}.}\it}{}
\newenvironment{bartproof}[1]{\noindent\textbf{Proof of~#1.}}{\smallskip}
\newenvironment{bartdef}[1]{\noindent\begin{minipage}{\textwidth}\bigskip\begin{
definition}\textnormal{\textbf{#1}} \\ \hbra \begin{center} \begin{minipage}{\gn
at}}{\end{minipage}\end{center} \hket\end{definition}\medskip\end{minipage}}



%
% ABSTRACT
%
\newenvironment{abstract}%
{
	\cleardoublepage%
	%\thispagestyle{empty}%
	\thispagestyle{plain}%
	\null 
	\vfill
	\begin{center}%S
	\bfseries \abstractname 
	\end{center}
}%
{
	\vfill
	\null
}


\newcommand{\lineno}[1]{{\tt\codefont {\textcolor{ForestGreen}{#1}}}}
\newcommand{\incode}[1]{\texttt{#1}}

\newcommand{\keywordColor}[1]{\textcolor{keyword}{#1}}
\newcommand{\codeColor}[1]{\textcolor{Black}{#1}}
\newcommand{\typeColor}[1]{\textcolor{NavyBlue}{#1}}
\newcommand{\typeGColor}[1]{\textcolor{Orange}{#1}}

\newcommand{\incodeBlack}[1]{\codeColor{\incode{#1}}}
\newcommand{\incodeType}[1]{\typeColor{\incode{#1}}}
\newcommand{\incodeTypeG}[1]{\typeGColor{\incode{#1}}}
\newcommand{\incodeKeyword}[1]{\keywordColor{\incode{\textbf{#1}}}}

\newcommand{\generic}[2][]{
	\incode{<\incodeTypeG{#2}\ifempty{#1}{}{ \incode{extends} \incodeTypeG{#1}}>}
}

\newcommand{\incodeMethod}[1]{\textcolor{BrickRed}{\incode{#1}}}

\newcommand{\niconote}[1]{\textcolor{Fuchsia}{nico-note: #1}}


\newcommand{\error}[1]{\textcolor{BrickRed}{#1}}
\newcommand{\warning}[1]{\textcolor{BurntOrange}{#1}}
\newcommand{\info}[1]{\textcolor{NavyBlue}{#1}}

\crefname{appendix}{appendix}{appendices}
\Crefname{appendix}{Appendix}{Appendices}

%
% colors
%
\definecolor{LightGrey}{rgb}{0.95,0.95,0.95}
\definecolor{keyword}{HTML}{7F0055}


\newcommand{\inputJava}[1]{%[backgroundcolor=LightGrey]	
\begin{mdframed}
	\inputminted[%
	fontsize=\footnotesize,%
	%frame=single,%
	%framesep=1.5\fboxsep,%
	%rulecolor=\color{listingFrame}%
	]{java}{#1}%
	%\vspace{-2mm}%
\end{mdframed}
}%

\newcommand{\inputJavaLineos}[1]{%
\begin{mdframed}
	%\vspace{1mm}%
	\inputminted[%
	linenos,
	fontsize=\footnotesize,%
	%frame=single,%
	%framesep=1.5\fboxsep,%
	%rulecolor=\color{listingFrame}%
	]{java}{#1}%
	%\vspace{-2mm}%
\end{mdframed}
}%


% listing CO2 %
\newcommand{\inputCoco}[1]{
	\lstinputlisting[
		language=coco,
		numbers=none
	]{#1}
}
\newcommand{\inputCocoFloat}[4][]{
	\begin{listing}
		\lstinputlisting[
			language=coco,
			numbers=left,
%			float=#1,
%			caption={#3},
%			label={#4},
%			captionpos=b
		]{#2}
		\caption{#3}
		\label{#4}
	\end{listing}
}
\newcommand{\inlineCoco}[1]{\lstinline[language=coco,basicstyle=\normalsize\ttfamily]|#1|}

% listing MAUDE %
\newcommand{\inputMaude}[1]{
	\lstinputlisting[
		language=maude,
		numbers=none
	]{#1}
}