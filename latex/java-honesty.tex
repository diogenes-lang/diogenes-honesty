
\section{The Java honesty checker}\label{sec:java-honesty}

\begin{itemize}
\item transformation from \coco process into java 
\item \coco plugin
\item honesty checker
\end{itemize}


\subsection{From \coco to jva }
The \coco Eclipse plugin is implemented with Xtext, a framework for development of programming languages and \textit{Domain Specific Languages} (DSL) (\Cref{sec:xtext}), which moreover is part of the Eclipse ecosystem and maintained as an official project. The correct way of define our plugin would be \textit{Xtext plugin}: in fact we create an Xtext project, which provides all the scaffolding classes needed to add special functionality like code validation, syntax highlighting, etc.


We define the \coco grammar using the Xtext DSL. Since the generated parser is LL(*), it is simple to write a grammar that fulfils the \coco specifications (although we must careful to not abuse of \textit{backtracking} due to performance reasons). The basic idea is that each rule corresponds to a Java class (instantiated when building the AST) and each declaration inside a rule is mapped to a class field. The implemented Xtext grammars are shown on \Cref{appchap:xtext-grammars}, categorized by type.
