\begin{abstract}
  Developing distributed applications typically requires
  to integrate new code with legacy third-party services,
  \eg, e-commerce facilities, maps, \etc %
  These services cannot always be assumed to smoothly collaborate with each other;
  rather, they live in a ``wild'' environment
  where they must compete for resources,
  and possibly diverge from the expected behaviour
  if they find it convenient to do so. %
  To overcome these issues,
  some recent works have proposed to discipline the interaction
  of mutually distrusting services through \emph{behavioural contracts}. %
  The idea is a dynamic composition, 
  where only those services with \emph{compliant} contracts 
  can establish sessions through which they interact. %
  Compliance between contracts guarantees that,
  if services behave honestly, they will enjoy safe interactions. %
  %
  The interaction between two participant is \emph{monitored} by a \emph{contract-oriented middleware} which can sanction a culprit that does not respect the contract it advertised, decreasing its \emph{reputation}: the lower is the reputation,
  the lower is the probability of being able to establish new sessions with it. This mechanism allows new form of attacks and a crucial problem is how to avoid these when deploying a service.

  A crucial property of contract-oriented systems if \emph{honesty}.
  A service is honest if it always respects the contract it advertises,
  in all possible execution contexts. %
  Being honest is important, because otherwise a service 
  can be blamed and sanctioned by the middleware. %
  Currently, honesty can be checked only at the specification level, 
  but there is not any guarantee that the corresponding implementation 
  preserves the property.
  
  We develop an Eclipse plugin that allows to write a formal specification of a contract-oriented application, which is automatically compiled 
  to a Java implementation that exploits the above-mentioned middleware.
  %
  We propose a new technique to verify honesty directly on the Java implementation.
  %
  Overall, our work allows to reduce the time spent on developing 
  contract-oriented services, and to make them more secure.
  % , as well as the risk of incurring in sanctions.
\end{abstract}