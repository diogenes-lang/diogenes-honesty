\begin{abstract}
  Modern distributed applications are typically obtained
  by integrating new code with legacy (and possibly untrusted)
  third-party services.
  Some recent works have proposed to discipline the interaction
  among these services through \emph{behavioural contracts}. %
  The idea is a dynamic discovery and composition of services,
  where only those with compliant contracts can interact,
  and their execution is monitored to detect and sanction 
  contract breaches.
  %
  In this setting, a service is said \emph{honest} 
  if it always respects the contract it advertises.
  Being honest is crucial, because it guarantees a service 
  not to be sanctioned; %
  further, compositions of honest services enjoy deadlock-freedom. %
  %
  However, developing honest programs is not an easy task,
  because contracts must be respected even in the presence of
  failures (whether accidental or malicious) of the context.
  In this paper we present Diogenes, a suite of tools 
  which supports programmers in writing honest Java programs.
  %
  Through an Eclipse plugin, programmers can
  write a \coco specification of the service,
  verify its honesty, and translate it into a skeletal Java program.
  Then, they can refine this skeleton into proper Java code,
  and use the tool to verify that its honesty has not been compromised
  by the refinement.
\end{abstract}

%  Currently, honesty can be checked only at the specification level, 
%  but there is not any guarantee that the corresponding implementation 
%  preserves the property.
%  We propose a new technique to verify honesty directly on the Java implementation.
%  We develop  that allows to write a formal specification of a contract-oriented application, which is automatically compiled 
%  to a Java implementation that exploits the above-mentioned middleware.
%  Overall, our work allows to reduce the time spent on developing 
%  contract-oriented services, and to make them more secure.
