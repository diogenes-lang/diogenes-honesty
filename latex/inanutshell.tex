\section{Diogenes in a nutshell}

In this section we show the main features of our tools.
%
\hidden{Firstly, we give an overview of the possible \coco processes
that you can write within Eclipse. 
Then, we present Diogenes, our verification tool for java programs.}



\paragraph{Contracts.}
A contract describes the intended behaviour of a
participant involved in a session.  We use first-order binary session
types~\cite{Honda98esop} as contracts. 
% Session types are terms
%of a process algebra featuring internal/external choice, and
%recursion.

Suppose we want to model a store which waits for an \atom{order},
then sends either the corresponding \atom{amount} or an \atom{abort} message.
The answer may depend on an external service, \eg an insurance company,
which waits for a \atom{req}est, then it answers \atom{ok} or \atom{no}.
%
In Diogenes, we write these two contracts as follows:
%
\begin{lstlisting}[language=coco,basicstyle=\scriptsize\ttfamily]
contract C { order? string . ( amount! int (+) abort! ) }
contract D { req! string . ( ok? + no? ) }
\end{lstlisting}
Receive actions are followed by the question mark (\code{?}) and grouped
with the symbol \code{+}; similarly, we use the
exclamation mark (\code{!}) and the symbol \code{(+)} for
send actions. If omitted, the type of an action is \code{unit}.

\hidden{The corresponding java contract is
\begin{mdframed}
\begin{minted}[
    fontsize=\scriptsize
    %,linenos
    ]{java}
ContractDefinition C = def("C").setContract(
    externalSum().add("order", Sort.string(), internalSum()
                                              .add("amount", Sort.integer())
                                              .add("abort")));     
ContractDefinition D = def("D").setContract(
    internalSum().add("req", Sort.string(), externalSum().add("ok").add("no")));
\end{minted}
\end{mdframed}
where \code{def()}, \code{externalSum()}, \code{internalSum()} 
are static methods of a factory.
}

\paragraph{Specification.}
A naïve \coco specification of our store is the following
\begin{lstlisting}[
    language=coco,
    basicstyle=\scriptsize\ttfamily,
    numbers=left,
    numbersep=12pt]
specification Pdishonest {
    tellAndWait x C .
    receive x [
        order? v:string . (
            tellAndWait y D . (
                send y req! v .
                receive y [
                    ok? . send x amount! 100
                    + no? . send x abort!
                    + t . send x abort!]))]
}
\end{lstlisting}

The primitive \inlineCoco{tellAndWait} at line \lineno{2}
advertises the contract \code{C},
and then waits until the session is established
When this happens, the variable \code{x} is bound to the session name.
Then, the store waits to receive an \atom{order}, 
binding it to the variable \code{v} (lines \lineno{3}-\lineno{4}).
Then, the store advertises the contract \code{D}, waits to establish a session
\code{y} (line \lineno{5}) and send a \atom{req}uest (line \lineno{6}) with
the value \code{v}.
Finally, the store waits to receive a response \atom{ok} or \atom{no},
and responds \atom{amount} or \atom{abort} on \code{x}. 
The action \inlineCoco{t} models a timeout 
that covers the case of no response (lines \lineno{7}-\lineno{10}).

Our tool correctly detecteds that the above specification is \emph{dishonest}.
Indeed, if the session \code{y}
is never established, 
the store is stuck at line \lineno{5} 
and cannot fulfil the contract \code{C} at session \code{x}.
Furthermore, what happens if the a response arrive after the timeout?
The store does not consume the input and 
therefore does not respect the contract \code{D}.

A possible way to fix the above specification is the following:
\begin{lstlisting}[
    language=coco,
    basicstyle=\scriptsize\ttfamily,
    numbers=left,
    numbersep=12pt]
specification Phonest {
    tellAndWait x C .
    receive x [
        order? v:string . (
            tellRetract y D . (
                send y req! v .
                receive y [
                    ok? . send x amount! 100
                    + no? . send x abort!
                    + t . (send x abort! | receive y [ok? + no?])]) 
            : send x abort!)]
}
\end{lstlisting}
At line \lineno{5} we use the primitive \inlineCoco{tellRetract}
which ensures that if the session \code{x} is not established
within a given deadline (immaterial in the specification)
the contract \code{D} is retracted,
and the control passes to line \lineno{11}.
Finally we change the specification of the timeout branch,
adding a parallel execution of a \inlineCoco{receive}
to consume possible inputs.

\hidden{The specification does not aim to be completed. On refining the implementation,
the user must provide suitable timeout values, and 
the order's amount should be derived in some way.}

\hidden{
The \cref{fig:outline} shows the outline derived from the above specification.

\begin{figure}[H]
\centering
\includegraphics[scale=0.4]{img/outline.png}
\label{fig:outline}
\caption{caption}
\end{figure}
}

\paragraph{Code generation.}
The plugin automatically generates corresponding Maude and Java files.
The former can be used directly within Eclipse to verify its honesty,
exploiting the model-checking tool of~\cite{verifiable};
the latter represents a working java skeleton. 
Considering the honest specification describe above,
Diogenes generates a skeleton that looks like as follows
\begin{mdframed}
\begin{minted}[
    fontsize=\scriptsize
    ,linenos
    ]{java}
public class Phonest extends Participant { 
   public void run() {
      Session<TST> x = tellAndWait(C);           // tellAndWait x C
       
      Message msg = x.waitForReceive("order");   // receive c [ order? v:string ]
      String v = msg.getStringValue();
      
      try {
         Session<TST> y = tellAndWait(D, 10000); // tellRetract y D
         y.sendIfAllowed("req", v);
         
         try {                                   // receive y [ok? + no? + t]
            Message msg_1 = y.waitForReceive(10000, "ok", "no");
            switch (msg_1.getLabel()) {                    
               case "ok": x.sendIfAllowed("amount", 100); break;
               case "no": x.sendIfAllowed("abort"); break;                    
            }
         }
         catch (TimeExpiredException e) {        // send x abort! | receive y [ok? + no?] 
            parallel(()->{ x.sendIfAllowed("abort"); });
            parallel(()->{ y.waitForReceive("ok", "no"); });
         }            
      }
      catch(ContractExpiredException e) {
         x.sendIfAllowed("abort");               // : send x abort (line 11)
      }
   }
}
\end{minted}
\end{mdframed}

\hidden{The \inlineCoco{tellRetract} corresponds to perform a tell with
a delay (line \lineno{10}). The successive \code{waitForSession()} blocks until the delay is
expired, throwing a \code{ContractExpiredException} if the session 
was not fused in time (the contract has been retracted).
The timeout \inlineCoco{t} corresponds to a \code{waitForReceived} with a timeout.
The method blocks until a message with label \code{"ok"} or \code{"no"} is received,
or throws a \code{TimeExpiredException} if the timeout expires.}

\paragraph{Refining the skeleton.}
The order amount is still hardcoded, so we decide to separate its computation
in a separated method, \eg
\begin{mdframed}
\begin{minted}[
    fontsize=\scriptsize
%    ,linenos
    ]{java}
public int getOrderAmount(String order) throws MyException {
    File f = new File("/orders/"+order);
    try (BufferedReader br = new BufferedReader( new FileReader(f) )) {
        return Integer.valueOf(br.readLine());        
    }
    catch (IOException e) { throw new MyException("Error: cannot read the order", e); }    
}
\end{minted}
\end{mdframed}
and change the number 100 in line \lineno{19} with \code{getOrderAmount(v)}.

\paragraph{Diogenes.}
Diogenes allows to verify the honesty of Java programs.
The honesty of a class extending \code{it.unica.co2.api.process.Participant} 
can be verified using the static method 
\code{HonestyChecker.isHonest(Class<? extends Participant>)}.

It returns the enumeration \code{HonestyResult} that can be
\begin{itemize}
\item \code{HONEST}: the tool was able to extract a \coco model and verify that it is honest;
\item \code{DISHONEST}: as above, but the model was dishonest;
\item \code{UNKNOWN}: the tool was unable to extract a model. 
It can be caused by several issues, such as an error of the 
tool or unhandled exceptions within the class under test.
\end{itemize}

Consider for example the refinement provided above.
What happens if the method throws \code{MyException} at runtime?
The user wants to be sure that its application covers all possible scenarios,
so we provide the annotation \code{@SkipMethod (value="<value>")}.
Diogenes treats annotated methods in this way:
1) retrieves all the declared exceptions;
2) inspects the return type;
3) symbolically considers both the case of a normal termination of the method
(returning the value specified in the annotation, if provided),
and all the possible exceptional terminations, 
independently from the method's instructions.

Annotating the code as follows
\begin{mdframed}
\begin{minted}[
    fontsize=\scriptsize
%    ,linenos
    ]{java}
@SkipMethod
public int getOrderAmount(String order) throws MyException { ... }
\end{minted}
\end{mdframed}

and invoking \code{HonestyChecker.isHonest(Phonest.class)},
Diogenes answers \code{UNKNOWN} and prints the following information
\begin{mdframed}
\begin{minted}[
    fontsize=\scriptsize
%    ,linenos
    ]{java}
error details: MyException: 
    This exception is thrown by the honesty checker. Please catch it!
        at it.unica.co2.store.Store$Phonest.getOrderAmount(Store.java:166)
        at it.unica.co2.store.Store$Phonest.run(Store.java:129)
        at it.unica.co2.honesty.HonestyChecker.runProcess(HonestyChecker.java:182)
\end{minted}
\end{mdframed}
Finally, adding \code{x.sendIfAllowed("abort")} when catching the exception,
the program come back \emph{honest}.






